\documentclass{article}
\usepackage[utf8]{inputenc}
\usepackage{amsmath}
\usepackage{amssymb}
%paquetes que usa el doc

\title{Hoja de trabajo \# 2}
\author{Luis Gerardo Morales Salazar \\Carnet: 2018-1364\\ morales181364@unis.edu.gt}
\date{02 de agosto de 2018}

% datos del encabezado o titulo del doc
\usepackage{natbib}
\usepackage{graphicx}


\begin{document}
\maketitle
% bloque de respuestas del ejercicio 2
\section{Ejercicio \# 1}

%respuestas del inciso 1
\begin{enumerate}
\item Demostrar usando inducci\'on lo siguiente $\\ \forall\ n.\ n^3\geq n^2$
\\\\Caso base:\\ $n=0 \\ 0^3\geq 0^2 \\ 0\geq 0$ 
\\\\Hipotesis Inductiva: \\ $ n^3\geq n^2\ $ 
\\\\Demostraci\'on :
\\$ n^3\geq n^2\ \\ n*n^2\geq n^2\ \\ (1+n)*(1+n)^2\geq (n+1)^2\ \\ n+1\geq \frac{(n+1)^2}{(n+1)^2}\ \\ n+1\geq 1 \\ n\geq 1-1 \\ n\geq 0 \ $ \end{enumerate}

% bloque de respuestas del inciso 2
\section{Ejercicio \# 2}
\begin{enumerate}
\item Demostrar utilizando inducci\'on la desigualdad de Bernoulli lo siguiente $\\ \forall\ n.\ (1+x)^n\geq nx $
\\donde $n\in \mathbb{N}$, $x\in \mathbb{Q}$ y $x\geq -1$
\\\\Caso base: $ n=0 \\ (1+x)^0\geq (0)x \\ 1\geq 0 $
    
\\\\Hipotesis Inductiva: $ (1+x)^n\geq nx $ 
\\\\Demostraci\'on cuando x es positiva :
\\ $(1+x)^{(n+1)} \geq x(n+1) \\ (1+x)(1+x)^n\geq x(n+1) \\ (1+x)^n+x(1+x)^n\geq (nx+x) \\  x(n+1)^n\geq x \\ (n+1)^n \geq 1 \\ nx \geq 1$

\\\\ Demostración cuando x es negativo:
\\ $(1+x)(1+x)^{(n+1)}\leq x(n+1) \\ (1+x)nx\leq x(n+1) \\ (1+x)n\leq n+1 \\n+nx\leq n+1 \\ nx\leq 1 \\ Si -1 \leq x\leq 0 \\ entonces: nx\leq 1 $ 
    
% documento creado en sharelatex.com
\end{enumerate}
\end{document}